\documentclass[12pt,a4paper,twoside]{article}
\usepackage[T1]{fontenc}
\usepackage[utf8]{inputenc}
\usepackage[polish]{babel}
\usepackage{graphicx}
\usepackage{times}
\usepackage{indentfirst}
\usepackage[left=3cm,right=2cm,top=2.5cm,bottom=2.5cm]{geometry}
\usepackage{natbib}
\usepackage{enumitem}
\usepackage{color}
\usepackage{tikz}
\usepackage{booktabs}
\usepackage{tabulary}
\setlist{itemsep=0pt}
\setlist{nolistsep}
\frenchspacing
\linespread{1.5}
\addto\captionspolish{%
\renewcommand*\listtablename{Spis tabel}
\renewcommand*\tablename{Tabela}
}
\usepackage{titlesec}
\titlelabel{\thetitle.\quad}
\usepackage{etoolbox}
\makeatletter
\patchcmd{\ttlh@hang}{\parindent\z@}{\parindent\z@\leavevmode}{}{}
\patchcmd{\ttlh@hang}{\noindent}{}{}{}
\makeatother

\begin{document}

\begin{center}

  \includegraphics[scale=0.3]{sgh_full.png}

  \vspace{1cm}
  Studium magisterskie

\end{center}

\vspace{1cm}

\noindent Kierunek: Analiza danych - big data

\noindent Specjalność: \dots

\vspace{1cm}

{
\leftskip=10cm\noindent
Roni Chikhmous\newline
Nr albumu: 69684

}

\vspace{2cm}

\begin{center}
  \LARGE
  Optymalizacja kosztowa procesu konstruowania głębokich sieci neuronowych (ang. deep neural networks) z wykorzystaniem chmury obliczeniowej
\end{center}

\vspace{1cm}

{
\leftskip=10cm\noindent
Praca magisterska napisana\newline
w Instytucie Ekonometrii\newline
pod kierunkiem naukowym\newline
dr. Przemysława Szufla

}

\vfill

\begin{center}
Warszawa, 2017
\end{center}
\thispagestyle{empty}

\clearpage
\thispagestyle{empty}
\mbox{}

% druga strona będzie pusta, ponieważ drukujemy dwustronnie
\clearpage

\tableofcontents

\clearpage

\section{Tytuł pracy}

Optymalizacja kosztowa procesu konstruowania glębokich sieci neuronowych (ang. deep neural networks) z wykorzystaniem chmury obliczeniowej

%\clearpage

\section{Problem badawczy}

Przedmiotem badania są jednostki obliczeniowe dostępne w ramach serwisów oferujących przetwarzanie danych w chmurze (ściślej – ich wydajność i stosunek wydajność/cena).

Celem pracy jest ocena jak duże w sensie oszczędności czasowej są korzyści płynące z wykorzystywania GPU w uczeniu skomplikowanych sieci neuronowych i znalezienie optymalnego kosztowo rozwiązania dla owego problemu. Ponadto, badanie ma na celu odpowiedzieć na pytanie, czy owe oszczędności czasowe potrzebne na uczenie i predykcję kompensują w wystarczającym stopniu wyższą cenę wynajmu tego typu jednostek obliczeniowych?

%\clearpage

\section{Dlaczego problem jest ważny dla ekonomisty?}

Wykorzystywanie GPU do szybszego uczenia modeli sieci neuronowych jest rozwiązaniem zyskującym na popularności. Praca ma na celu odpowiedzenie na pytanie, czy w przypadku pewnych problemów biznesowych, optymalniejsze kosztowo jest wynajmowanie tego typu dedykowanych instancji, czy korzystanie z tańszych jednostek.

Problem był poruszany w literaturze, jednakże porównywane były poszczególne GPU, nie jednostki oferowane przez platformy cloud computingowe.

%\clearpage

\section{Hipotezy badawcze}

Wsparcie procesu budowy modelu deep learningowego architekturą GPU prowadzi do mniejszego czasu potrzebnego na uczenie modelu i czasu na predykcję, jednakże ze względu na swoją popularność oraz wyższą cenę, jednostki zawierające GPU są rozwiązaniami mniej optymalnymi kosztowo.

%\clearpage

\section{Metody badawcze, wykorzystane dane}

Eksperymentalne porównanie wydajności dostępnych jednostek obliczeniowych (czasy uczenia sieci neuronowej uzyskany za pomocą wielokrotnych symulacji) oferowanych przez największych dostawców, w tym:
\begin{itemize}
\item AWS P2 (Accelerated Computing, general purpose GPU)
\item AWS C4 (Computing Optimized)
\item (?)Microsoft Azure NC/NV Series
\item (?)Google cloud instance with GPU

\end{itemize}

%\clearpage

\section{Plan pracy}

\begin{table}[h]
\centering
\caption{Plan ramowy pracy magisterskiej.}
\label{tab:planramowy}
\footnotesize
\begin{tabulary}{1.0\textwidth}{rLC}
\toprule
L.p. & Temat & Przewidywana liczba stron\\
\hline
1 & Wstęp. Ogólny zarys problemu i przedmiotu badań & 1 -- 2 \\
2.1 & Sieci neuronowe i deep learning - podstawy teoretyczne, zastosowania w rozpoznawaniu obrazu. Sposoby na przyspieszenie uczenia. & 9 -- 10 \\
2.2 & Opis rynku cloud computingu, przedstawienie platform oferujących usługi przetwarzania w chmurze & 11 -- 13 \\
3 & Przybliżenie porównywanych jednostek obliczeniowych, zbioru danych i oprogramowania.
Przedstawienie eksperymentu, metodologii przeprowadzania testu oraz wybranego algorytmu & 8 -- 9 \\
4 & Opis wyników eksperymentu, analiza rezultatów symulacji i przedstawienie wniosków & 12 -- 20 \\
5 & Podsumowanie pracy & 2 -- 3 \\
6 & Literatura & 2 -- 3 \\
7 & Spis tabel & 1 \\
8 & Spis rysunków & 1 \\
9 & Załączniki & 10 -- 15 \\
\hline
\multicolumn{2}{r}{$\sum$} & 57 -- 77 \\

\hline
\end{tabulary}
\end{table}

\clearpage


\end{document}
